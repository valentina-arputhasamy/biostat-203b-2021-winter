% Options for packages loaded elsewhere
\PassOptionsToPackage{unicode}{hyperref}
\PassOptionsToPackage{hyphens}{url}
%
\documentclass[
]{article}
\usepackage{lmodern}
\usepackage{amssymb,amsmath}
\usepackage{ifxetex,ifluatex}
\ifnum 0\ifxetex 1\fi\ifluatex 1\fi=0 % if pdftex
  \usepackage[T1]{fontenc}
  \usepackage[utf8]{inputenc}
  \usepackage{textcomp} % provide euro and other symbols
\else % if luatex or xetex
  \usepackage{unicode-math}
  \defaultfontfeatures{Scale=MatchLowercase}
  \defaultfontfeatures[\rmfamily]{Ligatures=TeX,Scale=1}
\fi
% Use upquote if available, for straight quotes in verbatim environments
\IfFileExists{upquote.sty}{\usepackage{upquote}}{}
\IfFileExists{microtype.sty}{% use microtype if available
  \usepackage[]{microtype}
  \UseMicrotypeSet[protrusion]{basicmath} % disable protrusion for tt fonts
}{}
\makeatletter
\@ifundefined{KOMAClassName}{% if non-KOMA class
  \IfFileExists{parskip.sty}{%
    \usepackage{parskip}
  }{% else
    \setlength{\parindent}{0pt}
    \setlength{\parskip}{6pt plus 2pt minus 1pt}}
}{% if KOMA class
  \KOMAoptions{parskip=half}}
\makeatother
\usepackage{xcolor}
\IfFileExists{xurl.sty}{\usepackage{xurl}}{} % add URL line breaks if available
\IfFileExists{bookmark.sty}{\usepackage{bookmark}}{\usepackage{hyperref}}
\hypersetup{
  pdftitle={Biostat 203B Homework 1},
  pdfauthor={Valentina Arputhasamy},
  hidelinks,
  pdfcreator={LaTeX via pandoc}}
\urlstyle{same} % disable monospaced font for URLs
\usepackage[margin=1in]{geometry}
\usepackage{color}
\usepackage{fancyvrb}
\newcommand{\VerbBar}{|}
\newcommand{\VERB}{\Verb[commandchars=\\\{\}]}
\DefineVerbatimEnvironment{Highlighting}{Verbatim}{commandchars=\\\{\}}
% Add ',fontsize=\small' for more characters per line
\usepackage{framed}
\definecolor{shadecolor}{RGB}{248,248,248}
\newenvironment{Shaded}{\begin{snugshade}}{\end{snugshade}}
\newcommand{\AlertTok}[1]{\textcolor[rgb]{0.94,0.16,0.16}{#1}}
\newcommand{\AnnotationTok}[1]{\textcolor[rgb]{0.56,0.35,0.01}{\textbf{\textit{#1}}}}
\newcommand{\AttributeTok}[1]{\textcolor[rgb]{0.77,0.63,0.00}{#1}}
\newcommand{\BaseNTok}[1]{\textcolor[rgb]{0.00,0.00,0.81}{#1}}
\newcommand{\BuiltInTok}[1]{#1}
\newcommand{\CharTok}[1]{\textcolor[rgb]{0.31,0.60,0.02}{#1}}
\newcommand{\CommentTok}[1]{\textcolor[rgb]{0.56,0.35,0.01}{\textit{#1}}}
\newcommand{\CommentVarTok}[1]{\textcolor[rgb]{0.56,0.35,0.01}{\textbf{\textit{#1}}}}
\newcommand{\ConstantTok}[1]{\textcolor[rgb]{0.00,0.00,0.00}{#1}}
\newcommand{\ControlFlowTok}[1]{\textcolor[rgb]{0.13,0.29,0.53}{\textbf{#1}}}
\newcommand{\DataTypeTok}[1]{\textcolor[rgb]{0.13,0.29,0.53}{#1}}
\newcommand{\DecValTok}[1]{\textcolor[rgb]{0.00,0.00,0.81}{#1}}
\newcommand{\DocumentationTok}[1]{\textcolor[rgb]{0.56,0.35,0.01}{\textbf{\textit{#1}}}}
\newcommand{\ErrorTok}[1]{\textcolor[rgb]{0.64,0.00,0.00}{\textbf{#1}}}
\newcommand{\ExtensionTok}[1]{#1}
\newcommand{\FloatTok}[1]{\textcolor[rgb]{0.00,0.00,0.81}{#1}}
\newcommand{\FunctionTok}[1]{\textcolor[rgb]{0.00,0.00,0.00}{#1}}
\newcommand{\ImportTok}[1]{#1}
\newcommand{\InformationTok}[1]{\textcolor[rgb]{0.56,0.35,0.01}{\textbf{\textit{#1}}}}
\newcommand{\KeywordTok}[1]{\textcolor[rgb]{0.13,0.29,0.53}{\textbf{#1}}}
\newcommand{\NormalTok}[1]{#1}
\newcommand{\OperatorTok}[1]{\textcolor[rgb]{0.81,0.36,0.00}{\textbf{#1}}}
\newcommand{\OtherTok}[1]{\textcolor[rgb]{0.56,0.35,0.01}{#1}}
\newcommand{\PreprocessorTok}[1]{\textcolor[rgb]{0.56,0.35,0.01}{\textit{#1}}}
\newcommand{\RegionMarkerTok}[1]{#1}
\newcommand{\SpecialCharTok}[1]{\textcolor[rgb]{0.00,0.00,0.00}{#1}}
\newcommand{\SpecialStringTok}[1]{\textcolor[rgb]{0.31,0.60,0.02}{#1}}
\newcommand{\StringTok}[1]{\textcolor[rgb]{0.31,0.60,0.02}{#1}}
\newcommand{\VariableTok}[1]{\textcolor[rgb]{0.00,0.00,0.00}{#1}}
\newcommand{\VerbatimStringTok}[1]{\textcolor[rgb]{0.31,0.60,0.02}{#1}}
\newcommand{\WarningTok}[1]{\textcolor[rgb]{0.56,0.35,0.01}{\textbf{\textit{#1}}}}
\usepackage{graphicx,grffile}
\makeatletter
\def\maxwidth{\ifdim\Gin@nat@width>\linewidth\linewidth\else\Gin@nat@width\fi}
\def\maxheight{\ifdim\Gin@nat@height>\textheight\textheight\else\Gin@nat@height\fi}
\makeatother
% Scale images if necessary, so that they will not overflow the page
% margins by default, and it is still possible to overwrite the defaults
% using explicit options in \includegraphics[width, height, ...]{}
\setkeys{Gin}{width=\maxwidth,height=\maxheight,keepaspectratio}
% Set default figure placement to htbp
\makeatletter
\def\fps@figure{htbp}
\makeatother
\setlength{\emergencystretch}{3em} % prevent overfull lines
\providecommand{\tightlist}{%
  \setlength{\itemsep}{0pt}\setlength{\parskip}{0pt}}
\setcounter{secnumdepth}{-\maxdimen} % remove section numbering

\title{Biostat 203B Homework 1}
\usepackage{etoolbox}
\makeatletter
\providecommand{\subtitle}[1]{% add subtitle to \maketitle
  \apptocmd{\@title}{\par {\large #1 \par}}{}{}
}
\makeatother
\subtitle{Due Jan 22 @ 11:59PM}
\author{Valentina Arputhasamy}
\date{}

\begin{document}
\maketitle

Display machine information for reproducibility:

\begin{Shaded}
\begin{Highlighting}[]
\KeywordTok{sessionInfo}\NormalTok{()}
\end{Highlighting}
\end{Shaded}

\hypertarget{q1.-gitgithub}{%
\subsection{Q1. Git/GitHub}\label{q1.-gitgithub}}

\textbf{No handwritten homework reports are accepted for this course.}
We work with Git and GitHub. Efficient and abundant use of Git, e.g.,
frequent and well-documented commits, is an important criterion for
grading your homework.

\begin{enumerate}
\def\labelenumi{\arabic{enumi}.}
\tightlist
\item
  Apply for the \href{https://education.github.com/pack}{Student
  Developer Pack} at GitHub using your UCLA email.
\end{enumerate}

\textbf{Solution: Done}

\begin{enumerate}
\def\labelenumi{\arabic{enumi}.}
\setcounter{enumi}{1}
\tightlist
\item
  Create a \textbf{private} repository \texttt{biostat-203b-2021-winter}
  and add \texttt{Hua-Zhou}, \texttt{Chris-German} and
  \texttt{ElvisCuiHan} as your collaborators with write permission.
\end{enumerate}

\textbf{Solution: Done}

\begin{enumerate}
\def\labelenumi{\arabic{enumi}.}
\setcounter{enumi}{2}
\tightlist
\item
  Top directories of the repository should be \texttt{hw1},
  \texttt{hw2}, \ldots{} Maintain two branches \texttt{master} and
  \texttt{develop}. The \texttt{develop} branch will be your main
  playground, the place where you develop solution (code) to homework
  problems and write up report. The \texttt{master} branch will be your
  presentation area. Submit your homework files (R markdown file
  \texttt{Rmd}, \texttt{html} file converted from R markdown, all code
  and data sets to reproduce results) in \texttt{master} branch.
\end{enumerate}

\textbf{Solution: Done}

\begin{enumerate}
\def\labelenumi{\arabic{enumi}.}
\setcounter{enumi}{3}
\tightlist
\item
  After each homework due date, teaching assistant and instructor will
  check out your master branch for grading. Tag each of your homework
  submissions with tag names \texttt{hw1}, \texttt{hw2}, \ldots{}
  Tagging time will be used as your submission time. That means if you
  tag your \texttt{hw1} submission after deadline, penalty points will
  be deducted for late submission.
\end{enumerate}

\textbf{Solution: Done}

\begin{enumerate}
\def\labelenumi{\arabic{enumi}.}
\setcounter{enumi}{4}
\tightlist
\item
  After this course, you can make this repository public and use it to
  demonstrate your skill sets on job market.
\end{enumerate}

\hypertarget{q2.-linux-shell-commands}{%
\subsection{Q2. Linux Shell Commands}\label{q2.-linux-shell-commands}}

\begin{enumerate}
\def\labelenumi{\arabic{enumi}.}
\item
  This exercise (and later in this course) uses the
  \href{https://mimic-iv.mit.edu}{MIMIC-IV data}, a freely accessible
  critical care database developed by the MIT Lab for Computational
  Physiology. Follow the instructions at
  \url{https://mimic-iv.mit.edu/docs/access/} to (1) complete the CITI
  \texttt{Data\ or\ Specimens\ Only\ Research} course and (2) obtain the
  PhysioNet credential for using the MIMIC-IV data. Display the
  verification links to your completion report and completion
  certificate here. (Hint: The CITI training takes a couple hours and
  the PhysioNet credentialing takes a couple days; do not leave it to
  the last minute.)
\item
  The \texttt{/usr/203b-data/mimic-iv/} folder on teaching server
  contains data sets from MIMIC-IV. Refer to
  \url{https://mimic-iv.mit.edu/docs/datasets/} for details of data
  files.

\begin{Shaded}
\begin{Highlighting}[]
\FunctionTok{ls}\NormalTok{ -l /usr/203b-data/mimic-iv}
\end{Highlighting}
\end{Shaded}

\begin{verbatim}
## total 12
## drwxr-xr-x. 2 huazhou huazhou   78 Jan 11 02:28 core
## drwxr-xr-x. 2 huazhou huazhou 4096 Jan 11 02:13 hosp
## drwxr-xr-x. 2 huazhou huazhou  189 Jan 11 02:25 icu
## -rw-r--r--. 1 huazhou huazhou 2518 Jan 11 02:13 LICENSE.txt
## -rw-r--r--. 1 huazhou huazhou 2459 Jan 11 02:13 SHA256SUMS.txt
\end{verbatim}

  Please, do \textbf{not} put these data files into Git; they are big.
  Do \textbf{not} copy them into your directory. Do \textbf{not}
  decompress the gz data files. These create unnecessary big files on
  storage and are not big data friendly practices. Just read from the
  data folder \texttt{/usr/203b-data/mimic-iv} directly in following
  exercises.

  Use Bash commands to answer following questions.
\item
  Display the contents in the folders \texttt{core}, \texttt{hosp},
  \texttt{icu}. What are the functionalities of the bash commands
  \texttt{zcat}, \texttt{zless}, \texttt{zmore}, and \texttt{zgrep}?
\item
  What's the output of following bash script?

\begin{Shaded}
\begin{Highlighting}[]
\KeywordTok{for} \ExtensionTok{datafile}\NormalTok{ in /usr/203b-data/mimic-iv/core/*.gz}
  \KeywordTok{do}
    \FunctionTok{ls}\NormalTok{ -l }\VariableTok{$datafile}
  \KeywordTok{done}
\end{Highlighting}
\end{Shaded}

  Display the number of lines in each data file using a similar loop.
\item
  Display the first few lines of \texttt{admissions.csv.gz}. How many
  rows are in this data file? How many unique patients (identified by
  \texttt{subject\_id}) are in this data file? What are the possible
  values taken by each of the variable \texttt{admission\_type},
  \texttt{admission\_location}, \texttt{insurance}, \texttt{language},
  \texttt{marital\_status}, and \texttt{ethnicity}? Also report the
  count for each unique value of these variables. (Hint: combine Linux
  commands \texttt{zcat}, \texttt{head}/\texttt{tail}, \texttt{awk},
  \texttt{uniq}, \texttt{wc}, and so on.)
\end{enumerate}

\hypertarget{q3.-whos-popular-in-price-and-prejudice}{%
\subsection{Q3. Who's popular in Price and
Prejudice}\label{q3.-whos-popular-in-price-and-prejudice}}

\begin{enumerate}
\def\labelenumi{\arabic{enumi}.}
\item
  You and your friend just have finished reading \emph{Pride and
  Prejudice} by Jane Austen. Among the four main characters in the book,
  Elizabeth, Jane, Lydia, and Darcy, your friend thinks that Darcy was
  the most mentioned. You, however, are certain it was Elizabeth. Obtain
  the full text of the novel from
  \url{http://www.gutenberg.org/cache/epub/42671/pg42671.txt} and save
  to your local folder.

\begin{Shaded}
\begin{Highlighting}[]
\ExtensionTok{curl}\NormalTok{ http://www.gutenberg.org/cache/epub/42671/pg42671.txt }\OperatorTok{>}\NormalTok{ pride_and_prejudice.txt}
\end{Highlighting}
\end{Shaded}

  Do \textbf{not} put this text file \texttt{pride\_and\_prejudice.txt}
  in Git. Using a \texttt{for} loop, how would you tabulate the number
  of times each of the four characters is mentioned?
\item
  What's the difference between the following two commands?

\begin{Shaded}
\begin{Highlighting}[]
\BuiltInTok{echo} \StringTok{'hello, world'} \OperatorTok{>}\NormalTok{ test1.txt}
\end{Highlighting}
\end{Shaded}

  and

\begin{Shaded}
\begin{Highlighting}[]
\BuiltInTok{echo} \StringTok{'hello, world'} \OperatorTok{>>}\NormalTok{ test2.txt}
\end{Highlighting}
\end{Shaded}
\item
  Using your favorite text editor (e.g., \texttt{vi}), type the
  following and save the file as \texttt{middle.sh}:

\begin{Shaded}
\begin{Highlighting}[]
\CommentTok{#!/bin/sh}
\CommentTok{# Select lines from the middle of a file.}
\CommentTok{# Usage: bash middle.sh filename end_line num_lines}
\FunctionTok{head}\NormalTok{ -n }\StringTok{"}\VariableTok{$2}\StringTok{"} \StringTok{"}\VariableTok{$1}\StringTok{"} \KeywordTok{|} \FunctionTok{tail}\NormalTok{ -n }\StringTok{"}\VariableTok{$3}\StringTok{"}
\end{Highlighting}
\end{Shaded}

  Using \texttt{chmod} make the file executable by the owner, and run

\begin{Shaded}
\begin{Highlighting}[]
\ExtensionTok{./middle.sh}\NormalTok{ pride_and_prejudice.txt 20 5}
\end{Highlighting}
\end{Shaded}

  Explain the output. Explain the meaning of \texttt{"\$1"},
  \texttt{"\$2"}, and \texttt{"\$3"} in this shell script. Why do we
  need the first line of the shell script?
\end{enumerate}

\hypertarget{q4.-more-fun-with-linux}{%
\subsubsection{Q4. More fun with Linux}\label{q4.-more-fun-with-linux}}

Try these commands in Bash and interpret the results: \texttt{cal},
\texttt{cal\ 2021}, \texttt{cal\ 9\ 1752} (anything unusual?),
\texttt{date}, \texttt{hostname}, \texttt{arch}, \texttt{uname\ -a},
\texttt{uptime}, \texttt{who\ am\ i}, \texttt{who}, \texttt{w},
\texttt{id}, \texttt{last\ \textbar{}\ head},
\texttt{echo\ \{con,pre\}\{sent,fer\}\{s,ed\}}, \texttt{time\ sleep\ 5},
\texttt{history\ \textbar{}\ tail}.

\textbf{Solution} :

\texttt{cal} displays a calendar of the current month.

\begin{Shaded}
\begin{Highlighting}[]
\FunctionTok{cal}
\end{Highlighting}
\end{Shaded}

\begin{verbatim}
##     January 2021    
## Su Mo Tu We Th Fr Sa
##                 1  2
##  3  4  5  6  7  8  9
## 10 11 12 13 14 15 16
## 17 18 19 20 21 22 23
## 24 25 26 27 28 29 30
## 31
\end{verbatim}

\texttt{cal\ 2021} displays a calendar for each month of the year 2021.

\begin{Shaded}
\begin{Highlighting}[]
\FunctionTok{cal}\NormalTok{ 2021}
\end{Highlighting}
\end{Shaded}

\begin{verbatim}
##                                2021                               
## 
##        January               February                 March       
## Su Mo Tu We Th Fr Sa   Su Mo Tu We Th Fr Sa   Su Mo Tu We Th Fr Sa
##                 1  2       1  2  3  4  5  6       1  2  3  4  5  6
##  3  4  5  6  7  8  9    7  8  9 10 11 12 13    7  8  9 10 11 12 13
## 10 11 12 13 14 15 16   14 15 16 17 18 19 20   14 15 16 17 18 19 20
## 17 18 19 20 21 22 23   21 22 23 24 25 26 27   21 22 23 24 25 26 27
## 24 25 26 27 28 29 30   28                     28 29 30 31
## 31
##         April                   May                   June        
## Su Mo Tu We Th Fr Sa   Su Mo Tu We Th Fr Sa   Su Mo Tu We Th Fr Sa
##              1  2  3                      1          1  2  3  4  5
##  4  5  6  7  8  9 10    2  3  4  5  6  7  8    6  7  8  9 10 11 12
## 11 12 13 14 15 16 17    9 10 11 12 13 14 15   13 14 15 16 17 18 19
## 18 19 20 21 22 23 24   16 17 18 19 20 21 22   20 21 22 23 24 25 26
## 25 26 27 28 29 30      23 24 25 26 27 28 29   27 28 29 30
##                        30 31
##         July                  August                September     
## Su Mo Tu We Th Fr Sa   Su Mo Tu We Th Fr Sa   Su Mo Tu We Th Fr Sa
##              1  2  3    1  2  3  4  5  6  7             1  2  3  4
##  4  5  6  7  8  9 10    8  9 10 11 12 13 14    5  6  7  8  9 10 11
## 11 12 13 14 15 16 17   15 16 17 18 19 20 21   12 13 14 15 16 17 18
## 18 19 20 21 22 23 24   22 23 24 25 26 27 28   19 20 21 22 23 24 25
## 25 26 27 28 29 30 31   29 30 31               26 27 28 29 30
## 
##        October               November               December      
## Su Mo Tu We Th Fr Sa   Su Mo Tu We Th Fr Sa   Su Mo Tu We Th Fr Sa
##                 1  2       1  2  3  4  5  6             1  2  3  4
##  3  4  5  6  7  8  9    7  8  9 10 11 12 13    5  6  7  8  9 10 11
## 10 11 12 13 14 15 16   14 15 16 17 18 19 20   12 13 14 15 16 17 18
## 17 18 19 20 21 22 23   21 22 23 24 25 26 27   19 20 21 22 23 24 25
## 24 25 26 27 28 29 30   28 29 30               26 27 28 29 30 31
## 31
\end{verbatim}

\end{document}
